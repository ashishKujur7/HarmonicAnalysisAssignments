\documentclass[12pt]{article}
\usepackage[margin=1in]{geometry}
\usepackage{amsfonts, amsmath}
\usepackage[T1]{fontenc}
\usepackage{mathrsfs, enumitem}
\usepackage{hyperref}
\usepackage[utf8]{inputenc}
\usepackage{amssymb}
\usepackage{amsfonts}
\usepackage{amsmath}
\usepackage{amsthm,cancel}
\usepackage{color}
\usepackage{hyperref}
\usepackage{csquotes}
%\usepackage{cmbright}
\usepackage{tikz-cd}
\usepackage{lipsum}
\usepackage{biblatex}
\addbibresource{ref.bib}
%\usepackage{stmaryrd}

\newtheorem{theorem}{Theorem}[section]
\newtheorem{lemma}[theorem]{Lemma}
\newtheorem{claim}[theorem]{Claim}
\newtheorem{proposition}[theorem]{Proposition}
\newtheorem{corollary}[theorem]{Corollary}
\newtheorem{fact}[theorem]{Fact}
\newtheorem{notation}[theorem]{Notation}
\newtheorem{observation}[theorem]{Observation}
\newtheorem{conjecture}[theorem]{Conjecture}
\newtheorem{exercise}[theorem]{Exercise}
\newtheorem{question}[theorem]{Question}

\theoremstyle{definition}
\newtheorem{definition}[theorem]{Definition}
\newtheorem{example}[theorem]{Example}
\numberwithin{equation}{section}

\theoremstyle{remark}
\newtheorem{remark}[theorem]{Remark}
\theoremstyle{plain}
\newcommand{\ignore}[1]{}

% section symbol
%\renewcommand{\thesection}{\S\arabic{section}}

% \renewcommand{\Pr}{{\bf Pr}}
% \newcommand{\Prx}{\mathop{\bf Pr\/}}
% \newcommand{\E}{{\bf E}}
% \newcommand{\Ex}{\mathop{\bf E\/}}
% \newcommand{\Var}{{\bf Var}}
% \newcommand{\Varx}{\mathop{\bf Var\/}}
% \newcommand{\Cov}{{\bf Cov}}
% \newcommand{\Covx}{\mathop{\bf Cov\/}}

% shortcuts for symbol names that are too long to type
\newcommand{\eps}{\epsilon}
\newcommand{\lam}{\lambda}
\renewcommand{\l}{\ell}
\newcommand{\la}{\langle}
\newcommand{\ra}{\rangle}
\newcommand{\wh}{\widehat}
\newcommand{\wt}{\widetilde}

% % "blackboard-fonted" letters for the reals, naturals etc.
\newcommand{\R}{\mathbb R}
\newcommand{\N}{\mathbb N}
\newcommand{\Z}{\mathbb Z}
\newcommand{\F}{\mathbb F}
\newcommand{\Q}{\mathbb Q}
\newcommand{\C}{\mathbb C}
\newcommand{\D}{\mathbb D}
\newcommand{\T}{\mathbb T}

% % operators that should be typeset in Roman font
% \newcommand{\poly}{\mathrm{poly}}
% \newcommand{\polylog}{\mathrm{polylog}}
% \newcommand{\sgn}{\mathrm{sgn}}
% \newcommand{\avg}{\mathop{\mathrm{avg}}}
% \newcommand{\val}{{\mathrm{val}}}

% % complexity classes
% \renewcommand{\P}{\mathrm{P}}
% \newcommand{\NP}{\mathrm{NP}}
% \newcommand{\BPP}{\mathrm{BPP}}
% \newcommand{\DTIME}{\mathrm{DTIME}}
% \newcommand{\ZPTIME}{\mathrm{ZPTIME}}
% \newcommand{\BPTIME}{\mathrm{BPTIME}}
% \newcommand{\NTIME}{\mathrm{NTIME}}

% values associated to optimization algorithm instances
\newcommand{\Opt}{{\mathsf{Opt}}}
\newcommand{\Alg}{{\mathsf{Alg}}}
\newcommand{\Lp}{{\mathsf{Lp}}}
\newcommand{\Sdp}{{\mathsf{Sdp}}}
\newcommand{\Exp}{{\mathsf{Exp}}}

% if you think the sum and product signs are too big in your math mode; x convention
% as in the probability operators
\newcommand{\littlesum}{{\textstyle \sum}}
\newcommand{\littlesumx}{\mathop{{\textstyle \sum}}}
\newcommand{\littleprod}{{\textstyle \prod}}
\newcommand{\littleprodx}{\mathop{{\textstyle \prod}}}

% horizontal line across the page
\newcommand{\horz}{
\vspace{-.4in}
\begin{center}
\begin{tabular}{p{\textwidth}}\\
\hline
\end{tabular}
\end{center}
}

% calligraphic letters
\newcommand{\calA}{{\cal A}}
\newcommand{\calB}{{\cal B}}
\newcommand{\calC}{{\cal C}}
\newcommand{\calD}{{\cal D}}
\newcommand{\calE}{{\cal E}}
\newcommand{\calF}{{\cal F}}
\newcommand{\calG}{{\cal G}}
\newcommand{\calH}{{\cal H}}
\newcommand{\calI}{{\cal I}}
\newcommand{\calJ}{{\cal J}}
\newcommand{\calK}{{\cal K}}
\newcommand{\calL}{{\cal L}}
\newcommand{\calM}{{\cal M}}
\newcommand{\calN}{{\cal N}}
\newcommand{\calO}{{\cal O}}
\newcommand{\calP}{{\cal P}}
\newcommand{\calQ}{{\cal Q}}
\newcommand{\calR}{{\cal R}}
\newcommand{\calS}{{\cal S}}
\newcommand{\calT}{{\cal T}}
\newcommand{\calU}{{\cal U}}
\newcommand{\calV}{{\cal V}}
\newcommand{\calW}{{\cal W}}
\newcommand{\calX}{{\cal X}}
\newcommand{\calY}{{\cal Y}}
\newcommand{\calZ}{{\cal Z}}

% bold letters (useful for random variables)
%----------------------------------------------
% \renewcommand{\a}{{\boldsymbol a}}
% \renewcommand{\b}{{\boldsymbol b}}
% \renewcommand{\c}{{\boldsymbol c}}
% \renewcommand{\d}{{\boldsymbol d}}
% \newcommand{\e}{{\boldsymbol e}}
% \newcommand{\f}{{\boldsymbol f}}
% \newcommand{\g}{{\boldsymbol g}}
% \newcommand{\h}{{\boldsymbol h}}
% \renewcommand{\i}{{\boldsymbol i}}
% \renewcommand{\j}{{\boldsymbol j}}
% \renewcommand{\k}{{\boldsymbol k}}
% \newcommand{\m}{{\boldsymbol m}}
% \newcommand{\n}{{\boldsymbol n}}
% \renewcommand{\o}{{\boldsymbol o}}
% \newcommand{\p}{{\boldsymbol p}}
% \newcommand{\q}{{\boldsymbol q}}
% \renewcommand{\r}{{\boldsymbol r}}
% \newcommand{\s}{{\boldsymbol s}}
% \renewcommand{\t}{{\boldsymbol t}}
% \renewcommand{\u}{{\boldsymbol u}}
% \renewcommand{\v}{{\boldsymbol v}}
% \newcommand{\w}{{\boldsymbol w}}
% \newcommand{\x}{{\boldsymbol x}}
% \newcommand{\y}{{\boldsymbol y}}
% \newcommand{\z}{{\boldsymbol z}}
% \newcommand{\A}{{\boldsymbol A}}
% \newcommand{\B}{{\boldsymbol B}}
% \newcommand{\C}{{\boldsymbol C}}
% \newcommand{\D}{{\boldsymbol D}}
% \newcommand{\E}{{\boldsymbol E}}
% \newcommand{\F}{{\boldsymbol F}}
% \newcommand{\G}{{\boldsymbol G}}
% \renewcommand{\H}{{\boldsymbol H}}
% \newcommand{\I}{{\boldsymbol I}}
% \newcommand{\J}{{\boldsymbol J}}
% \newcommand{\K}{{\boldsymbol K}}
% \renewcommand{\L}{{\boldsymbol L}}
% \newcommand{\M}{{\boldsymbol M}}
% \renewcommand{\O}{{\boldsymbol O}}
% \renewcommand{\P}{{\mathbb{P}}}
% \newcommand{\Q}{{\boldsymbol Q}}
% \newcommand{\R}{{\boldsymbol R}}
% \renewcommand{\S}{{\boldsymbol S}}
% \newcommand{\T}{{\boldsymbol T}}
% \newcommand{\U}{{\boldsymbol U}}
% \newcommand{\V}{{\boldsymbol V}}
% \newcommand{\W}{{\boldsymbol W}}
% \newcommand{\X}{{\boldsymbol X}}
% \newcommand{\Y}{{\boldsymbol Y}}
% \newcommand{\Z}{{\boldsymbol Z}}

% script letters
\newcommand{\scrA}{{\mathscr A}}
\newcommand{\scrB}{{\mathscr B}}
\newcommand{\scrC}{{\mathscr C}}
\newcommand{\scrD}{{\mathscr D}}
\newcommand{\scrE}{{\mathscr E}}
\newcommand{\scrF}{{\mathscr F}}
\newcommand{\scrG}{{\mathscr G}}
\newcommand{\scrH}{{\mathscr H}}
\newcommand{\scrI}{{\mathscr I}}
\newcommand{\scrJ}{{\mathscr J}}
\newcommand{\scrK}{{\mathscr K}}
\newcommand{\scrL}{{\mathscr L}}
\newcommand{\scrM}{{\mathscr M}}
\newcommand{\scrN}{{\mathscr N}}
\newcommand{\scrO}{{\mathscr O}}
\newcommand{\scrP}{{\mathscr P}}
\newcommand{\scrQ}{{\mathscr Q}}
\newcommand{\scrR}{{\mathscr R}}
\newcommand{\scrS}{{\mathscr S}}
\newcommand{\scrT}{{\mathscr T}}
\newcommand{\scrU}{{\mathscr U}}
\newcommand{\scrV}{{\mathscr V}}
\newcommand{\scrW}{{\mathscr W}}
\newcommand{\scrX}{{\mathscr X}}
\newcommand{\scrY}{{\mathscr Y}}
\newcommand{\scrZ}{{\mathscr Z}}

\newcommand{\im}{{\text{im }}}
\newcommand{\ip}[1]{\left\langle #1 \right\rangle}
\newcommand{\norm}[1]{\left\lVert #1 \right\rVert}
\newcommand{\abs}[1]{\left\lvert #1 \right\rvert}
\newcommand{\defbox}[1]{\fbox{\textsc{ #1 }}}
\newcommand{\coker}{\operatorname{coker}}
\newcommand{\ind}{\operatorname{ind}}
\newcommand{\supp}{\operatorname{supp}}
\newcommand{\dou}{\partial}
\newcommand{\Mult}{\operatorname{Mult}}

\newcommand\blfootnote[1]{%
  \begingroup
  \renewcommand\thefootnote{}\footnote{#1}%
  \addtocounter{footnote}{-1}%
  \endgroup
}

\title{Solutions to Euclidean Harmonic Analysis Assignment 2}
\author{Ashish Kujur --- PHD231027}

\date{\today}

\begin{document}
\maketitle 
\tableofcontents

\section{Question 1}
\horz
Suppose that ($X$, $\mathcal{M}$, $\mu$) and ($Y$, $\mathcal{N}$, $\nu$) are $\sigma$-finite measure spaces, and let $f$ be a product measurable function on $X \times Y.$ If $f \ge 0$ and $1 \le p < \infty$, then
\begin{equation}
 \left[\int_X \left(\int_Y f(x,y) d\nu(y) \right)^pd\mu(x)\right]^\frac{1}{p} \le \int_Y \left[\int_X f(x,y)^p d\mu(x)\right]^\frac{1}{p}d\nu(y)
 \label{eqn:ntp}
 \end{equation}
 If $1 \le p \le \infty$, $f(\cdot, y) \in L^p(\mu)$ for a.e. $y$, and the function $y \to ||f(\cdot, y)||_p$ is in $L^1(\nu)$, then $f(x, \cdot) \in L^1(\nu)$ for a.e. $x$, the function $x \to \int f(x,y) d\nu(y)$ is in $L^p(\mu)$, and $$\left|\left|\int f(\cdot, y)d\nu(y)\right|\right|_p \le \int||f(\cdot, y)||_pd\nu(y).$$  

To prove the above you can do it by the following steps.
\begin{itemize}
\item[(a)] Let $1\leq p<\infty$ and $g\in L_p(X, \mathcal{M}, \mu)$. Then $\|g\|_p=\{\int_{X}gh:\|h\|_q=1\}$ where $\frac{1}{p}+\frac{1}{q}=1.$ (Hint: Use Hölder's inequality and then consider the function $h(x)=|g(x)|^{q-1}\frac{\operatorname{sgn} \ g(x)}{\|g\|_q^{q-1}}$ where $\operatorname{sgn}\ g(x):=\frac{g(x)}{|g(x)|}$ when $g(x)\neq 0$ and $\operatorname{sgn} \ g(x)=0$ otherwise.)
\item[(b)] For $1<p<\infty,$ use (a) and Fubini's theorem to prove 1.
\end{itemize}
\horz

\begin{proof}[Solution]

To prove this question, we use the following result due to Tonelli:

\begin{proposition}[Tonelli]
Let $\left( X, \scrA, \mu \right)$ and $\left( Y, \scrB , \nu \right)$ be $\sigma$-finite measure spaces, and let $f: X \times Y \to [0, + \infty]$ be $\scrA \times \scrB$-measurable. Then 
\begin{enumerate}[label=(\alph*)]
\item the function $x \mapsto \int_{Y} f\left( x,y \right) d\nu \left( y \right)$ is $\scrA$-measurable and the function $y \mapsto \int_{X} f\left( x,y \right) d\mu \left( x \right)$ is $\scrB$-measurable, and 
\item $f$ satisfies
\begin{equation*}
\int_{X\times Y} f d\left( \mu \times \nu  \right) = \int_{X} \left( \int_{Y} f\left( x,y \right) d\nu \left( y \right) \right)d\mu \left( x \right) = \int_{Y} \left( \int_{X} f\left( x,y \right) d\mu \left( x \right) \right) d\nu \left( y \right)
\end{equation*}
\end{enumerate}
\label{thm:tonelli}
\end{proposition}

The statement and proof of the above proposition can be found in \cite{MR3098996} in Chapter 5, Section 2.

Also, we appeal to the following result which can be found in Theorem 6.16. in  \cite{MR0344043} and also in the proof of duality of $L^{p}$ in Proposition 3.5.5. in \cite{MR3098996}.

\begin{theorem}
Suppose $1 \le p <\infty$, $\mu$ is a $\sigma$-finite positive measure on $X$ and $\Phi$ is a bounded linear functional on $L^{p} \left( \mu \right)$. Then there is a unique $g \in L^{q} \left( \mu \right)$ where $q$ is the conjugate exponent of $p$ such that 
\begin{equation*}
\Phi \left( f \right) =\int_{X} fg \, d \mu \; \left( f\in L^{p} \left( \mu \right) \right).
\end{equation*}
Moreover, if $\Phi$ and $g$ are related as in the previous equaiton then we have
\begin{equation*}
\norm{\Phi} = \norm{g}_{q}.
\end{equation*}
\label{thm:duality}
\end{theorem}

We can start the proof now. Let $f$ be a measurable function on $X \times Y$. We first do it for the case where $f\ge 0$. For $p=1$, we are done because then it is Proposition \ref{thm:tonelli} in disguise. So, suppose that $1<p< \infty$. Also, we may assume $\int_{X} \left( \int_{Y} f(x,y)^{p} d\mu (x) \right)^{1/p} d\nu (y) < \infty$ for otherwise the inequality \ref{eqn:ntp} is always true.


Now, define $F\left( x \right) = \int_{Y} f(x,y) d\nu \left( y \right)$. By Proposition \ref{thm:tonelli}, $F$ is $\scrM$-measurable.

Now, we define a linear functional $\Phi : L^{q} \left( \mu \right) \to \C$ by the following way:
\begin{equation*}
\Phi (g) =\int_{X} gF \; d\mu \left( g \in L^{q} \; \left( \mu \right) \right)
\end{equation*}

We now show that $\Phi$ is bounded linear functional. To this end, let $g \in L^{q} \left( \mu \right)$ and consider the following:

\begin{align*}
\abs{\int_{X} gF d\mu} &\le \int_{Y} \abs{g(x)} \abs{F(x)} d\mu (x)  \\
&\le \int_{X} \abs{g(x)} \left( \int_{Y} f(x,y) d\nu (y) \right) d\mu (x) \\
&\le \int_{X} \left( \int_{Y} f(x,y) \abs{g(x)} d\nu (y) \right) d\mu (x) \\
&\le \int_{X} \left( \int_{Y} f(x,y) \abs{g(x)} d\mu (x) \right) d\nu (y) & \left( \text{Proposition } \ref{thm:tonelli} \right) \\
&\le \int_{X} \left( \int_{Y} f(x,y)^{p} d\mu (x) \right)^{1/p} \norm{g}_{L^{q} \left( \mu \right)} d\nu (y) & \left( \text{Holder's inquality} \right) \\
&\le \norm{g}_{L^{q} \left( \mu \right)}  \underbrace{\int_{X} \left( \int_{Y} f(x,y)^{p} d\mu (x) \right)^{1/p} d\nu (y)}_{< \infty \text{ by assumption}}.
\end{align*}

By Theorem \ref{thm:tonelli}, we have that there is some unique $h \in L^{p} \left( \mu \right)$ such that $\Phi (g) = \int_{X} hg d\mu$ for each $g \in L^{q} \left( \mu \right)$.

By uniqueness, we also have that $F=h$ $\mu$-almost everywhere. Hence $F$ is in $L^{p} \left( \mu \right)$. It follows that $\norm{F}_{L^{p} \left( \mu \right)} =  \left[\int_X \left(\int_Y f(x,y) d\nu(y) \right)^pd\mu(x)\right]^\frac{1}{p} \le \int_{X} \left( \int_{Y} f(x,y)^{p} d\mu (x) \right)^{1/p} d\nu (y)$.

Now, we proceed to complete the second part. Let $f$ be a measurable function on $X \times Y$ such that $f\left( \cdot, y \right) \in L^{p} \left( \mu \right)$ for $\nu$-almost every $y$ and the function $y \mapsto \norm{f\left( \cdot , y \right)}_{p}$ is in $L^{1} \left( \nu \right)$. First, consider the case where $1\le p < \infty$.

Since the function $y \mapsto \norm{f\left( \cdot , y \right)}_{p}$ is in $L^{1} \left( \nu \right)$, we have that
\begin{equation*}
\int_{X} \left( \int_{Y} \abs{f(x,y)}^{p} d\mu (x) \right)^{1/p} d\nu (y) < \infty.
\end{equation*}
So, we repeat the above proof taking $f$ as $\abs{f}$ (as all the hypothesis are met) and we have that the function $x \to \int f(x,y) d\nu(y)$ is in $L^p(\mu)$ and
\begin{equation*}
\left[\int_X \left(\int_Y \abs{f(x,y)} d\nu(y) \right)^pd\mu(x)\right]^\frac{1}{p} \le \int_{X} \left( \int_{Y} \abs{f(x,y)}^{p} d\mu (x) \right)^{1/p} d\nu (y).
\end{equation*}
which is the same as
$$\left|\left|\int f(\cdot, y)d\nu(y)\right|\right|_p \le \int||f(\cdot, y)||_pd\nu(y).$$ 
To finish the proof for $p=\infty$. Fix $y\in Y$. Hence for $\mu$-almost every $x$,
\begin{equation*}
f\left( x,y \right) \le \norm{f\left( \cdot, y \right)}_{\infty}.
\end{equation*}
It follows that for each $x \in X$,
\begin{equation*}
\int_{Y} f\left( x,y \right) d\nu \left( y \right) \le \int_{Y} \norm{f\left( \cdot, y \right)} d\nu \left( y \right)
\end{equation*}
Taking (essential)-supremum over $x\in X$, we have
\begin{equation*}
\norm{\int_{Y} f\left( x,y \right) d\nu \left( y \right)}_{\infty} \le \int_{Y} \norm{f\left( \cdot, y \right)}_{\infty} d\nu \left( y \right).
\end{equation*}
\end{proof}

\section{Assignment Question 2}
\horz
Let $D_{N}$ be the Dirichlet kernel. Prove that
\begin{equation*}
\frac{4}{\pi^{2}} \sum_{k=1}^{N} \frac{1}{k} \le \norm{D_{N}}_{L^{1} (\T)} \le 2 + \frac{\pi}{4} + \frac{4}{\pi ^{2}} \sum_{k=1}^{N} \frac{1}{k}.
\end{equation*}
\horz

\begin{proof}[Solution]
First, we prove the lower bound estimate for $\norm{D_{N}}_{L^{1}\left( \T \right)}$. We have already shown that
\begin{equation*}
D_{N} \left( t \right) = \frac{\sin \left(\left( 2N+1 \right)\pi t\right)}{\sin \left( \pi t \right)}.
\end{equation*}
To estimate the lower bound, consider the following:
\begin{align*}
\norm{D_{N}}_{L^{1} \left( \T \right)} &= \int_{0}^{1} \abs{ \frac{\sin \left( \left( 2N+1 \right)\pi t \right)}{\sin \left( \pi t \right)}} \, dt \\
&= 2 \int_{0}^{\frac{1}{2}}  \abs{ \frac{\sin \left( \left( 2N+1 \right)\pi t \right)}{\sin \left( \pi t \right)}} \, dt & \text{(symmetric about } x=1/2 ) \\
&= \frac{2}{\pi} \int_{0}^{\frac{\pi}{2}} \abs{\frac{\sin \left( \left( 2N +1 \right)t \right)}{\sin t}} \, dt & \text{(substitute } u=\pi t ) \\
& \ge   \frac{2}{\pi} \int_{0}^{\frac{\pi}{2}} \abs{\frac{\sin \left( \left( 2N +1 \right)t \right)}{t}} \, dt  \\
& =   \frac{2\left( 2N+1 \right)}{\pi} \int_{0}^{\frac{(2N+1)\pi}{2}} \abs{\frac{\sin \left( t \right)}{t}} \, dt   & \left(\text{substitute }u=( 2N+1 )t \right)  \\
&\ge \frac{2}{\pi} \int_{0}^{\frac{(2N+1)\pi}{2}} \abs{\frac{\sin \left( t \right)}{t}} \, dt   \\
&= \frac{2}{\pi} \left( \sum_{k=0}^{n-1} \int_{k\pi}^{\left( k+1 \right)\pi} \frac{\abs{\sin t}}{t} \, dt + \int_{(k+1)\pi}^{\frac{\left( 2N+1 \right)}{2}} \frac{\abs{\sin t}}{t} \, dt \right) \\
&\ge \frac{2}{\pi} \sum_{k=0}^{n-1} \int_{k\pi}^{\left( k+1 \right)\pi} \frac{\abs{\sin t}}{t} \, dt  \\
&\ge\frac{2}{\pi} \sum_{k=0}^{n-1} \int_{0}^{\pi} \frac{\sin t}{ t+k\pi} \, dt & \left( \text{substitute } u=t+k\pi \right) \\
&\ge\frac{2}{\pi} \sum_{k=0}^{n-1} \int_{0}^{\pi} \frac{\sin t}{ \pi+k\pi} \, dt  \\
&= \frac{2}{\pi ^{2}} \sum_{k=0}^{n-1} \frac{1}{k+1} \int_{0}^{\pi} \sin t \, dt \\
&= \frac{4}{\pi ^{2}} \sum_{k=1}^{n} \frac{1}{k}.
\end{align*}

To estimate the upper bound, consider the following:
\end{proof}

\section{Question 3}
\horz
Let $f\in L_1(\mathbb R).$ Define $\hat{f}(\zeta)=\int_{\mathbb R}f(x)e^{-2\pi ix\zeta}dx.$ Show that if $\int_{\mathbb{R}}|x||f(x)|dx<\infty,$ then we must have that $\hat{f}$ is continuously differentiable. Find a condition on $f$ for which $\hat{f}$ will be a smooth function.
\horz
\begin{proof}[Solution]
We need to show that the function $\hat{f}$ is continuously differentiable. Fix a point $\zeta_{0} \in \R$. Consider the sequence of function $F_n : \R \to \C$ given by
\begin{equation*}
F_{n} \left( x \right) = f\left( x \right)e^{-2\pi i x \zeta_{0}}\left( \frac{e^{\left( -2\pi i \frac{x}{n} \right)} - 1}{1/n} \right)
\end{equation*}
for each $x \in \R$.
It is easy to see that $F_{n} \left( x \right)$ converges pointwise everywhere to $-2\pi i x f(x)e^{-2\pi i x \zeta_{0}}$ because $\lim_{n\to \infty} \frac{e^{\left( -2\pi i \frac{x}{n} \right)} - 1}{1/n} = 0$ for each  $x \in \R$. 
Also, we claim that $\abs{F_n (x)} \le 2\pi \abs{x f(x)}$. To show this, we make use of the fact that $\abs{e^{ix} -1 } \le \abs{x}$ for each $x\in \R$. \footnote{One can show this by using that fact that $\abs{\sin x}\le \abs{x}$ for each  $x\in \R$ as follows:
\begin{align*}
\abs{e^{ix} - 1} &= \abs{\cos x + i \sin x -1} \\
&= \abs{1- \cos x + i \sin x} \\
&= \abs{2 \sin ^{2} \left( \frac{x}{2} \right) + 2i \sin \left( \frac{x}{2} \right) \cos \left( \frac{x}{2} \right)} \\
&= \abs{2i \sin \left( \frac{x}{2} \right) e^{\frac{ix}{2}}} \\
&\le 2 \frac{\abs{x}}{2} = \abs{x}.
\end{align*}}
To this end, let $x\in \R$ and consider the following:
\begin{align*}
\abs{F_{n}\left( x \right)} &= \abs{f\left( x \right)e^{-2\pi i x \zeta_{0}}\left( \frac{e^{\left( -2\pi i \frac{x}{n} \right)} - 1}{1/n} \right)} \\
&\le \abs{f(x)} \frac{\abs{-2\pi i \frac{x}{n}}}{1/n} \\
&= 2\pi \abs{xf(x)} 
\end{align*}

We went through the trouble of defining the sequence of function $\left( F_{n} \right)_{n\in \N}$ for the reason that:
\begin{align*}
\frac{d\hat{f}}{d\zeta} \left( \zeta_{0} \right) &= \lim_{n \to \infty} \frac{\hat{f} \left( \zeta_{0} + \frac{1}{n} \right) - \hat{f} \left( \zeta_{0} \right)}{\frac{1}{n}} \\
&= \lim_{n\to \infty} \int_{\R} F_{n} \left( x \right) \, dx \\
&= \int_{\R} \lim_{n\to \infty} F_{n} \left( x \right) \, dx & \text{(by DCT)} \\
&= \int_{\R}  -2\pi i x f(x)e^{-2\pi i x \zeta_{0}} \, dx \\
&= -2\pi i \widehat{\left( x f \right)} \left( \zeta_{0} \right).
\end{align*}

This shows that $\hat{f}$ is differentiable and $\frac{d\hat{f}}{d\zeta} \left( \zeta \right) = -2\pi i \widehat{\left( xf \right)} \left( \zeta \right)$. To see that this is continuous, let $\zeta_{0} \in R$ and let $\left( h_{n} \right)$ be any sequence converging to $0$ and we again apply DCT in the following way:
\begin{align*}
\lim_{n \to \infty} \widehat{xf} \left( \zeta_{0} + h_{n} \right) &= \lim_{n\to \infty} \int_{-\infty}^{\infty} x f\left( x \right) e^{-2\pi i \left(\zeta_{0} + h_{n} \right)} \\
&\stackrel{\text{(DCT)}} {=}  \int_{-\infty}^{\infty}  \lim_{n\to \infty}x f\left( x \right) e^{-2\pi i \left(\zeta_{0} + h_{n} \right)} \\
&= \int_{-\infty}^{\infty}  x f\left( x \right) e^{-2\pi i \zeta_{0}} \\
&= \widehat{xf} \left( \zeta_{0} \right)
\end{align*}
This would complete the proof provided we justify the step at DCT step. That is easily justified by the fact that $x \mapsto x f(x)$ is in $L^{1} \left( \R \right)$.

Finally, we showed in class that if $f\in \calS \left( \R \right)$, that is, the Schwartz class of $\R$ then $\hat{f}$ must be smooth.
\end{proof}

\section{Question 4}
\horz
 Let $f:\mathbb{R}\to\mathbb{C}$ be such that $\int_{\mathbb{R}}|f(x)|dx<\infty$ and $g\in C_c^\infty(\mathbb{R})$. Prove that the function $f*g$ defined by $f*g(x):=\int_{\mathbb{R}}f(y)g(x-y)dy$ is a well-defined smooth function.
\begin{itemize}
\item[(a)] Is $f*g$ will also be compactly supported? What if $f\in C_c^\infty(\mathbb R)$? 
\item[(b)] Prove that $\|f*g\|_p\leq \|f\|_1\|g\|_p$ for all $1\leq p\leq \infty.$
\end{itemize}
\horz

\begin{proof}[Solution]
\begin{enumerate}[label=(\alph*)]
\item  If $f \in L^{1} \left( \R \right)$ and $g \in C_{c} ^{\infty} \left( \R \right)$ then it can be shown that $f * g$ may not be compactly supported.

On the other hand, we show that if both $f$ and $g$ are both compactly supported then $f * g $ must be compactly supported. In fact, we show that if $\supp \left( f * g \right) \subset \overline{\supp f + \supp g}$. \footnote{The proof of the claim of $\supp \left( f *g \right) \subset \overline{\supp f + \supp g}$ of the solution was borrowed from \cite{MR2759829}.}.

First, we claim that for a fixed $x\in R$, $f\left( x-y \right) g\left( y \right) \ne 0$ implies that $y \in \left( x- \supp f \right) \cap \supp g$. To see this, let $y\in \R$ be such that $f\left( x-y \right) g(y) \ne 0$. Consequently, $f\left( x-y \right) \ne 0$ and $g\left( y \right) \ne 0$. If $g\left( y \right) \ne 0$ then $y \in \supp g$ and if $f\left( x-y \right) \ne 0$ then $x-y \in \supp f$, that is, $y \in x - \supp f$. Let $C_{x} = \left( x- \supp f \right) \cap \supp g$ for each $x\in \R$. 

Now, note that for a fixed $x\in \R$, we have that
\begin{align*}
\left( f * g \right) \left( x \right) &= \int_{\R} f\left( x-y \right) g(y) \, dy \\
&= \int_{C_{x}} f\left( x-y \right) g(y) \, dy + \cancelto{0}{\int_{C_{x}^{c}} f\left( x-y \right) g(y) \, dy} & (\text{by the previous paragraph})\\
&= \int_{\left( x- \supp f \right) \cap \supp g} f\left( x-y \right) g(y) \, dy.
\end{align*}

We claim that if $x \not \in \supp f + \supp g$ then $\left( x - \supp f \right) \cap \supp g = \emptyset$. This is to see for if $y \in \supp \left( x - \supp f \right) \cap \supp g$ then $x-y \in \supp f$ and $y \in \supp g$ which implies $x \in \supp f + \supp g$. 
Consequently, if $x \not \in \supp f + \supp g$ then we have that $\left( f * g \right)(x) = 0$. Hence, we have that $(f * g) = 0$ a.e. on $\left( \supp f + \supp g \right)^{c}$. Hence, $f * g = 0$ a.e. in particular on the interior of $\left( \supp f + \supp g \right)^{c}$ which equals $\overline{\supp f + \supp g}$.

Since sum of two compact sets is compact, we are done.\footnote{because sum is jointly continuous and product of compact sets is compact.}.
\item One can prove this with the weaker hypothesis that $g \in L^{p} \left( \R \right)$ in the following way:
\begin{align*}
\norm{f * g}_{p} &= \norm{\int_{\R} f(y) g\left( \cdot - y \right) \, dy}_{p} \\
&\le \int_{\R} \norm{f\left( y \right) g\left( \cdot - y \right)}_{p} \, dy & \text{(See question 1)} \\
&= \int_{\R} \abs{f(y)} \norm{g\left( \cdot - y \right)}_{p} \, dy \\
&= \int_{\R} \abs{f(y)} \norm{g}_{p} \, dy & \text{(translation invariance)} \\
&= \norm{f}_{1} \norm{g}_{p}.
\end{align*}
\end{enumerate}

\end{proof}

%\section{Question 5}
\horz

Let $P_r(t):=\frac{1-r^2}{1-2r\cos(2\pi t)+r^2}$ be the Poisson kernel where $0\leq r<1$ and $t\in\mathbb{R}.$ Show that for any $f\in L_p(\mathbb T)$ $P_r*f(x):=\int_{0}^1f(x-t)P_r(t)dt$ is well-defined.
\begin{itemize}
\item[(a)] Show that $t\mapsto P_r(t)$ is a non-negative twice continuously differentiable function for all $0\leq r<1.$
\item[(b)] Show that $\|P_r\|_{L_1(\mathbb{T})}=1$ and for all $0<\delta<\frac{1}{2},$ $\lim\limits_{r\to {1^{-}}}\int_{\delta<|t|<\frac{1}{2}}P_r(t)dt=0.$
\item[(c)] Show that for all $1\leq p<\infty$ and $f\in L_p(\mathbb{T}),$ we have $\lim\limits_{r\to {1^{-}}}\|P_r*f-f\|_{L_p(\mathbb{T})}=0.$ Show that the same is true when $p=\infty$ and $f\in C(\mathbb{T}).$
\item[(d)] Show that the function $u(re^{2\pi it}):=P_r*f(t)$ defines a map from $\mathbb{D}$ to $\mathbb{R}$ and it solves the Dirichlet problem i.e. $\frac{\partial^2u}{\partial x^2}+\frac{\partial^2u}{\partial y^2}=0$ in $\mathbb{D}.$ (Hint: Change everything in Polar coordinates and you need to show that $u$ is differentiable also)
\end{itemize}

\horz
\begin{proof}[Solution]
\begin{enumerate}[label=(\alph*)]
\item 
\end{enumerate}
\end{proof}


\nocite{*}
\printbibliography

\end{document}


