\section{Lecture Note Question 3}
\horz
\begin{enumerate}
\item Compute $\hat{f} \left( k \right)$ when 
\begin{equation*}
f(e^{2\pi i t}) = \sum_{k=-N}^{N} a_{k} e^{2\pi i k t}
\end{equation*}
\item Compute $\hat{f} \left( k \right)$ when 
\begin{equation*}
f(e^{2\pi i t}) = 
\begin{cases}
1 & \text{for } a<t<b \\
0 & \text{otherwise}
\end{cases}
\end{equation*}
where $[a,b] \subset [0,1)$.
\item Show that $\lim_{n\to \infty} \abs{\hat{f} (n)} = 0$ in the above example. 

\end{enumerate}

\horz

\begin{proof}[Solution]
\begin{enumerate}
\item A quick computation shows that 
\begin{equation*}
\hat{f} (n) = \begin{cases}
a_{k} & \text{for } -N \le k \le N \\
0 & \text{otherwise}.
\end{cases}
\end{equation*}
\item Again a simple computation shows that
\begin{equation*}
\hat{f} (n) = \frac{i}{2\pi n} \left( e^{-2\pi inb} - e^{-2\pi i n a} \right)
\end{equation*}
\item For the trignometric polynomial, the Fourier coefficients is eventually zero, hence, $\lim_{n\to \infty} \abs{\hat{f} (n)} = 0$. While for the second one, 
\begin{align*}
\abs{\hat{f} \left( n \right)} &= \abs{\frac{i}{2\pi n} \left( e^{-2\pi inb} - e^{-2\pi i n a} \right)} \\
& \le \frac{2}{2\pi n} = \frac{1}{\pi n} \to 0 \text { as } n \to \infty.
\end{align*}
\end{enumerate}
\end{proof}
